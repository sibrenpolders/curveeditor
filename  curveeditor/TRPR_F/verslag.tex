\documentclass[a4paper,11pt,oneside, titlepage]{article}
\author{ Groep 13: S. Staessens, S. Polders }
\title{\ \ \ \ \ \ Trimesteroverschrijdend Project:\newline Curve Editor}
\date{dinsdag 12 februari, 2008}
\usepackage[dutch]{babel}
\usepackage{verbatim}
\usepackage{graphicx}
\usepackage[colorlinks,urlcolor=blue,filecolor=magenta]{hyperref}
\usepackage{color}
\usepackage{url}
\parindent 0pt	
\hyphenation{}
\begin{document}
\maketitle\newpage
\section{Onderwerp: Curve Editor}
Het project delen we (voorlopig) op in de secties \it{basis} en \it{optioneel}. 
\rm{De basis geeft de functionaliteit weer die zeker in het project verwerkt zal worden. Optioneel
daarentegen zijn functies die pas ge\"implementeerd zullen worden als de basis grondig is 
uitgewerkt.}
\begin{itemize}
\item \bf{Basis}
\begin{itemize}
\item \rm{Punten ingeven ( m.b.v. muis of keyboard )}
\item Connectie van ingegeven punten ( Bezier, Hermite )
\item Verschillende curvens met elkaar connecteren
\item verplaatsen van punten ( m.b.v. muis of keyboard )
\item mogelijkheid om verschillende kleuren en diktes te gebruiken
\item Save \& load functies
\end{itemize}
\item \bf{Optioneel}
\begin{itemize}
\item \rm{Tools:}
\begin{itemize}
\item Soundmixer
\item Transformer
\item Curving
\item \ldots
\end{itemize}
\item Oplichting aangeklikte curve/punt
\item 3d curves
\item \ldots
\end{itemize}
\end{itemize}
\section{Algoritmes}
\begin{itemize}
\item \bf{Bezier:}\newline
\begin{enumerate}
\rm{ \item formule: $\sum_{i=0}^n (_i^n).P_i.(1-t)^{n-i}.t^i\ (\ t\ \epsilon\ [0,1]\ )$}
\item uitwerking:\newline
zij P een vector van controlepunten ( ingegeven punten ).\newline
zij B een vector met berekende punten.\newline
n het aantal ingegeven punten\newline
N het aantal punten tussen 0 en 1 dat we willen berekenen ( == t )\newline

initialiseer B met enkel 0-waarden.

\begin{tabbing}
from i \= = 0 to n do\ \textcolor{blue}{(1)}\\
\>from j \= = 1 to N do\\
\> \>from k \= = 0 to n do\\
\> \> \>B[i*N + j] += $(_k^n).P[k].(1-t)^{n-k}.t^k$\\
\\
Het \=algoritme doet het volgende:\\
\end{tabbing}
\end{enumerate}
\item \bf{Hermite}
\rm{ \begin{enumerate}
\item formule \& algoritme: \\
\begin{tabbing}
\\De vo\=lgende punten worden door de gebruik ingegeven:\\
\>P1: Het startpunt van de curve\\
\>T1: De richting waarin het beginpunt gaat ( de raaklijn van de curve in punt )\\
\>P2: Het eindpunt van de curve\\
\>T2: De richting waarin het eindpunt gaat ( de raaklijn van de curve in punt )\\
\\
Deze 4 punten worden dan vermenigvuldigd met de 4 basis functies van Hermite:\\
\>$h1(s) =  2s^3 - 3s^2 + 1$\\
\>$h2(s) = -2s^3 + 3s^2$\\
\>$h3(s) =   s^3 - 2s^2 + s$\\
\>$h4(s) =   s^3 -  s^2$\\
\\
Matrix S: Het interpolatie punt en zijn 3 machten\\
Matrix C: De parameters van de Hermite curve\\
Matrix h: De matrix notatie van de 4 Hermite functies\\ \\
$ S = \left( \begin{array}{c}
s^3 \\
s^2\\
s^1\\
1\\
\end{array} \right) $
$ C = \left( \begin{array}{c}
P1\\
P2\\
T1\\
T2\\          
\end{array} \right) $
$h = \left( \begin{array}{cccc}
2 & -2 & 1 & 1\\
-3 & 3 & -2 & -2\\
0 & 0 & 1 & 0\\
1 & 0 & 0 & 0\\          
\end{array} \right) $
\\ \\
Om een punt op de curve te berekenen stel je de vector S op, vermenigvuldig \\ je hem met de
matrices h en C.\\
$P = S * h * C$\\
\\
\end{tabbing}
\end{enumerate}
}
\item \bf{Curve zoeken met behulp van een gegeven punt}
\rm{}
\begin{enumerate}
\item Duid een punt aan op de curve die gezocht moet worden.
\item Doe zolang de curve niet gevonden is, voor elke volgende curve in de vector:
\item bereken voor elk punt in de curve de afstand tot dat punt
\item indien afstand $<$ 0.01 ga naar 5 anders 2
\item indien curve gevonden, ``Hoera!'', anders ``:-(``
\end{enumerate}
Mogelijke verfijning van het vorige algoritme:
\begin{enumerate}
\item Verdeel het tekengebied in N aantal gelijke rechtoeken
\item Verdeel deze rechthoeken over een goedgekoze hashmap
\item voeg een referenties van de curve toe bij de juist haskey
\item Bereken de afstand van de oorsprong tot het ingegeven punt
\item gebruik deze afstand om in de hasmap te zoeken welke curves punten hebben\\
op deze afstand van de oorsprong
\item ga met deze subset van curves naar stap 2 in het vorige algoritme
\end{enumerate}
Hierdoor zal het berekenen van een curve iets trager gaan ( de hasmap moet 
in orde gebracht worden ). Het zoeken daarentegen zal sneller gaan. Normaal gezien
is de gebruiker meer bereid om te wachten op een berekening dan te wachten 
op het vinden van een curve. Dus dit zou een goed overwogen tradeoff zijn.
\end{itemize}
\end{document}
