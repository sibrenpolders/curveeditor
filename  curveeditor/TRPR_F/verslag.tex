\documentclass[a4paper,11pt,oneside, titlepage]{article}
\author{ Groep 13: S. Staessens, S. Polders }
\title{\ \ \ \ \ \ Trimesteroverschrijdend Project:\newline Curve Editor}
\date{dinsdag 12 februari, 2008}
\usepackage[dutch]{babel}
\usepackage{verbatim}
\usepackage{graphicx}
\usepackage[colorlinks,urlcolor=blue,filecolor=magenta]{hyperref}
\usepackage{url}
\parindent 0pt	
\hyphenation{}
\begin{document}
\maketitle\newpage
\section{Onderwerp: Curve Editor}
Het project delen we (voorlopig) op in de secties \it{basis} en \it{optioneel}. 
\rm{De basis geeft de functionaliteit weer die zeker in het project verwerkt zal worden. Optioneel
daarentegen zijn functies die pas ge\"implementeerd zullen worden als de basis grondig is 
uitgewerkt.}
\begin{itemize}
\item \bf{Basis}
\begin{itemize}
\item \rm{Punten ingeven ( m.b.v. muis of keyboard )}
\item Connectie van ingegeven punten ( Bezier, Hermite )
\item Verschillende curvens met elkaar connecteren
\item verplaatsen van punten ( m.b.v. muis of keyboard )
\item mogelijkheid om verschillende kleuren en diktes te gebruiken
\item Save \& load functies
\end{itemize}
\item \bf{Optioneel}
\begin{itemize}
\item \rm{Tools:}
\begin{itemize}
\item Soundmixer
\item Transformer
\item Curving
\item \ldots
\end{itemize}
\item Oplichting aangeklikte curve/punt
\item 3d curves
\item \ldots
\end{itemize}
\end{itemize}
\end{document}
