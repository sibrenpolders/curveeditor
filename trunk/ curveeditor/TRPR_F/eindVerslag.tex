\documentclass[a4paper,11pt,oneside, titlepage]{article}
\author{Groep 13: Sibrand Staessens en Sibren Polders}
\title{Trimesteroverschrijdend Project: Curve Editor}
\date{Donderdag 22 mei, 2008}
\usepackage[dutch]{babel}
\usepackage{verbatim}
\usepackage{graphicx}
\usepackage[colorlinks,urlcolor=blue,filecolor=magenta]{hyperref}
\usepackage{url}
\parindent 0pt	
\hyphenation{Hermite deze Monitor-Pool}
\begin{document}
\maketitle \newpage
\tableofcontents \newpage
\section{Voorwoord}
Op welke manieren kan je een set van punten op een gladde wijze met elkaar verbinden? Dat was
de vraag achter Curve Editor. Het gemakkelijkste pad dat getekend kan worden tussen de punten
is natuurlijk lineair. Maar er zijn nog zoveel andere mogelijkheden, waarvan we er
slechts twee hebben verwerkt in curve editor (nl. Bezier en Hermite). Deze algorithmes zullen
vlakke kromme maken tussen de interpolatiepunten. De toepassingen van de algortitmes die we 
gebruikt hebben beperken zich niet enkel tot het tekenen van ``lijntjes'' tussen punten, 
maar ook op bijvoorbeeld camerabeweging of ``AI''-beweging in games.
De interpolatie tussen een gegeven set punten is een uitgebreide en interessante studie die 
op vele vlakken in de informatica/wiskunde zijn nut kan bewijzen. Curve editor geeft er de 
basistoepassing ( ofwel de toegepaste wiskundige ) ervan. \newline \newline \newline \newline
\newline \newline \newline \newline
\begin{center}
\textit{``If the path be beautiful, let us not ask where it leads.''} - Anatole France \newpage
\end{center}\newpage
\section{Wiskundige Voorkennis}
\subsection{lineair}
\subsection{Bezier}
\subsection{Hermite}
\newpage
\section{Implementatie}
\subsection{Packages}
\subsubsection{Java packages}
Een Java package is een mechanisme binnen Java om klassen te organiseren in namespaces. 
Java broncode die binnen eenzelfde categorie of functie vallen kunnen hierdoor gegroepeerd 
worden. Dit kan door middel van een package statement bovenaan het beginbestand om aan te 
geven waartoe ze behoren. Dit is om twee redenen handig omdat de sources gegroepeerd zijn
onder hun categorie, wat het geheel overzichtelijker maakt. Verder kunnen er nu twee 
verschillende klassen eenzelfde naam krijgen en toch uniek bepaald worden door er zijn
package name voor te zetten. Wat zeker handig is als de programmeur een klasse dezelfde naam
heeft gegeven als een klasse uit een library die hij wil gaan gebruiken.\newline \newline
Voor ons project hebben we een algemeen pakket src gemaakt die verschillende sub-pakketten bevat.
Wat duidelijk zichtbaar is in figuur BLAH.
In wat volgt wordt een korte beschrijving gegeven van al deze pakketten. Zonder al teveel in
te gaan op de technische details. 
\subsubsection{CurveEditor}
Dit is wellicht het kleinste pakket van de reeks (figuur BLAH). Het bevat slechts een klasse, 
namelijk de 
main klasse. Deze klasse zal, zoals wellicht duidelijk is, als bootstrap dienen voor de 
applicatie curve editor. Er wordt ook de mogenlijkheid geboden om al rechtstreeks vanuit
de commandline een file mee te geven. Dit is enkel ter volledigheid vermits de gebruiker tijdens
de loop van het programma zeer makkelijk bestanden kan inladen en opslaan.
\subsubsection{Algorithms}
Dit pakket voorziet allerlei klassens die voor de interpolatie tussen punten zullen zorgen.
Elke klasse in dit pakket implementeerd de Algorithme interface. Dit is handig voor de 
groepswerk vermits deze interface vastlegt welke functies de programmeur zal implementeren.
Zodoende weet je al op voorhand welke methodes je moet aanroepen om een bepaald resultaat
te verkrijgen. \newline
De klassenamen zijn triviaal gekozen: ``Lineair, Bezier, BezierC1, BezierG1, Hermite, 
Hermite Cardinaal, Hermite Catmull Rom''(figuur BLAH). 
Zoals de namen al verdaden zullen deze de verschillende
interpolatie methodes die besproken werden in het deel 'Wiskundige voorkennis' implementeren.
Hiervoor werd natuurlijk altijd geoogd op de meeste optimale implementatie van degene die 
besproken werden.
\subsubsection{Core}
Dit pakket bevat enkele noodzakkelijk klassens(figuur BLAH).\newline \newline
De klasse CurveContainer zal ervoor zorgen dat ingegeven punten kunnen opgeslagen worden, samen
met hun door interpolatie berekende punten. \newline \newline
Een eerste idee was het subdivsion principe toe te passen. In de beginsituatie is het tekenveld
dan een grote rechthoek. Van zodra de gebruiker een curve begint te tekenen worden de secties
waar punten geplaatst zijn onderverdeeld in steeds kleiner wordende rechthoekjes. In elk zo'n
rechthoekje zat dan juist een punt van een curve. Zodat er gemakkelijk gezegd kon worden
welk punt waar stond en tot welke curve het behoorde.\newline
Dit algorithme bleek echter niet zo effici\"ent te zijn wanneer we te maken hadden met een groot
aantal input punten. Dit kwam voornamelijk doordat er telkens opnieuwe een kleinere rechthoek
moest berekend worden bij ingeven van een nieuw punt. En een grotere wanneer er punten verwijderd
werden. Uiteindelijk was de applicatie meer bezig met het berekenen van rechthoekjes dan met zijn
doel: voorstellen van curves.\newline \newline
De tweede poging leek beter te lukken. We stelden een veld op waarvan elke pixel een mogenlijke
houder kon zijn van een punt. De houders werden ge\"initialiseerd met de waarde null zodat ze 
geen
plaats innamen. Het toevoegen van punten is zo simpel uit te voeren door het new commando toe
te passen. Verwijderen is dan gewoon de juist holder op null zetten ( de garbage collection van
java lost de rest op ). Het zoeken gaat simpelweg door de positie van de muisklik om te 
zetten naar co\"ordinaten die gebruikt kunnen worden op de vector waarin alle punten worden 
opgeslagen. Om daarna in een bepaalde vooraf bepaalde range te kijken of een punt houder niet
op null staat, is dat zo dan zal de informatie van dat punt teruggeven worden, zoniet is er op
die plaats in het veld geen punt beschikbaar.\newline \newline
De klasse Editor is het hart van de curve editor. Deze zal zorgen dat de data die uitgewisseld
moet worden kan en ook in de juiste richting zal stromen. De uitwisseling van data zal 
voornamelijk bestaan uit het zoeken of verwijderen van punten uit de CurveContainer klasse.
Maar ook het opvangen en afhandelen van excepties. Deze klasse zorgt er dus voor dat de 
verschillende andere klassen zo autonoom als mogenlijk kunnen werken. Dit verhoogd natuurlijk
enkel de leesbaarheid en onderhoudbaarheid van de code.\newline \newline
Een laatste klasse van dit pakket is de FileIO klasse, deze zal niet alleen files opslaan en
inladen, maar ook zal hij de funtonialiteit van undo en redo implementeren, vermits deze van
dezelfde functies gebruik maakt.
\subsubsection{Curve}
Dit pakket bevat de de twee datatypes die doorheen het programma gebruikt worden. Point zoals
de naam doet vermoeden geeft de mogenlijkheid een punt op te slaan. Curve geeft dan weer de
mogenlijkheid om een verzameling van punten ( lees een kromme of curve ) op te slaan. De
technische details van deze laatste klasse wordt beter uitgelegd verder in de tekst. 
Voorlopig is het voldoende om te weten dat Curve een vector van Point's bijhoudt en enkele
basisvoorzieningen voorziet ( punten opvragen, toevoeg, transleren, \ldots ).
\subsubsection{Exceptions}
Een foutloos programma schrijven is al een hele opgave, vermits er altijd wel kleine bugs
kunnen opduiken na langdurig gebruik. Een fool proof programma schrijven daarentegen is een
onmogelijke opgave. Daarom hebben we gebruik gemaakt van exceptions om ``verkeerd'' gebruik van
curve editor op te vangen. Onder verkeerd gebruik valt bijvoorbeeld het inladen van een
onbestaande file, of een verkeerd fileformat. Het toevoegen van een punt zonder er de 
co\"ordinaten van op te geven, \ldots.\newline \newline
Er zijn ook twee HandleException klassen voorzien. Eentje in dit pakket, deze zal gewoon 
het exceptie bericht in de console uitprinten. In het GUI pakket is een HandleException klasse
voorzien die in een dialog scherm de exceptie zal uitprinten.
\subsection{Uitwerking van de GUI}
\subsection{Java Listeners}
\subsection{Datastructuren}
\subsection{Extra's}
\newpage
\section{Planning}
\newpage
\section{Taakverdeling}
\newpage
\section{Appendix}
\appendix
\section{Handleiding}
\section{Screenshot}
\section{Referenties}
\end{document}
